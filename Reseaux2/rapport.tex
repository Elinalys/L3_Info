\documentclass[12pt, a4paper]{article}
\usepackage{lmodern}
\usepackage[utf8]{inputenc}
% \usepackage{fontspec}
\usepackage{amssymb,amsmath}
\usepackage[T1]{fontenc}
\usepackage[scale=0.8]{geometry} 
\usepackage[french]{babel}% pour un document en français.
\usepackage{minted}
\usepackage{graphicx}
\usepackage[colorlinks=true, linkcolor=blue]{hyperref}
\date{}
\title{Rapport de TP de Réseaux II}
\author{Emilie \textsc{Péret} \and Loris \textsc{Croce}}

\begin{document}

\maketitle

% \tableofcontents

\section*{Fonctionnement du protocole}



\section*{Technologie utilisée}

Pour ce TP, nous avons choisi d'utiliser le langage Javascript pour plusieurs raisons. D'une part son haut niveau permettait de s'abstraire de certaines contraintes telles que le typage ou la gestion de la mémoire. Mais c'est aussi pour ses propriétés inhérentes, en effet, fournir une interface sous la forme d'une page HTML simple nous a paru une solution efficace. De plus, Javascript permet de dessiner dans un \emph{canvas} déclaré dans la page.
Et meme si Javascript a pour défaut de nas as être uassi rapide que certains autres langages cet inconvenient était négligeable ici.
\section*{Implémentation}




\section*{Améliorations, remarques}

% taille pixel variable, correction d'erreur variable
\end{document}