\documentclass[]{article}
\usepackage{lmodern}
\usepackage{amssymb,amsmath}
\usepackage{ifxetex,ifluatex}
\usepackage{fixltx2e} % provides \textsubscript
\ifnum 0\ifxetex 1\fi\ifluatex 1\fi=0 % if pdftex
  \usepackage[T1]{fontenc}
  \usepackage[utf8]{inputenc}
\else % if luatex or xelatex
  \ifxetex
    \usepackage{mathspec}
  \else
    \usepackage{fontspec}
  \fi
  \defaultfontfeatures{Ligatures=TeX,Scale=MatchLowercase}
\fi
% % use upquote if available, for straight quotes in verbatim environments
% \IfFileExists{upquote.sty}{\usepackage{upquote}}{}
% % use microtype if available
% \IfFileExists{microtype.sty}{%
% \usepackage{microtype}
% \UseMicrotypeSet[protrusion]{basicmath} % disable protrusion for tt fonts
% }{}
% \usepackage{hyperref}
% \hypersetup{unicode=true,
%             pdfborder={0 0 0},
%             breaklinks=true}
% \urlstyle{same}  % don't use monospace font for urls
% \IfFileExists{parskip.sty}{%
% \usepackage{parskip}
% }{% else
% \setlength{\parindent}{0pt}
% \setlength{\parskip}{6pt plus 2pt minus 1pt}
% }
% \setlength{\emergencystretch}{3em}  % prevent overfull lines
% \providecommand{\tightlist}{%
%   \setlength{\itemsep}{0pt}\setlength{\parskip}{0pt}}
% \setcounter{secnumdepth}{0}
% % Redefines (sub)paragraphs to behave more like sections
% \ifx\paragraph\undefined\else
% \let\oldparagraph\paragraph
% \renewcommand{\paragraph}[1]{\oldparagraph{#1}\mbox{}}
% \fi
% \ifx\subparagraph\undefined\else
% \let\oldsubparagraph\subparagraph
% \renewcommand{\subparagraph}[1]{\oldsubparagraph{#1}\mbox{}}
% \fi

\date{}

\title{Études de cas}

\begin{document}


Ici, mini-problèmes : algorithme

\section{3 cas}\label{cas}

\begin{enumerate}
\def\labelenumi{\arabic{enumi}.}
\tightlist
\item
  Pb orienté optimisation et satisfaction des contraintes
\item
  Pb orienté ``simulation''
\item
  Pb orienté ``Apprentissag et réseaux de neurones''
\end{enumerate}

\section{Problème orienté
optimisation}\label{probluxe8me-orientuxe9-optimisation}

\subsection{Pb n reines}\label{pb-n-reines}

\begin{verbatim}
Input : Tableau n×n TAB a valeurs dans nombres entiers ̛̛supérieurs ou égaux à 0
    TAB[i,j] ...

Output : Un placement de n "reines" sur l'échiquier (marquage de n cases)

    tant que on a pas 2 cases marquées sur une même ligne, colonne, diagonales
    l'argent correspondant aux cases marquées soit maximal
\end{verbatim}

\paragraph{Étape 1}\label{uxe9tape-1}

Établier une spécification formelle via un modèle linéaire (PLNE).
\textbf{modèle linéaire} : calculer un vecteur Z a valeurs dans des
nombres (flottants ou entiers). tq (1) pour i dans J \textgreater{}
1..N, Z, doit etre entier (2) pour i dans 1..K, Σ Ak,i, , Zi ?? Bk (3)
Maximiser Σ Ci Zi

\begin{verbatim}
    Input : 
        N
        J <= (1..N)
        Ak, k = 1..K
        Bk, k = 1..K
        Ci, i = 1..N
\end{verbatim}

\textbf{Ce qu'il faut savoir} : * Les modèles linéaires → Beaucoup
d'effort des editeurs de logiciels - Bibliothèque dédiées IBM (OSL,
CPLEX), XPRESS, GUROBI - Bibliothèques de calcul généraliste MATLAB,
SAS, \ldots{} - Logiciels de gestion d'entreprises : SAP - Langages :
CHIP, Prolog, CP-optimizer * Ces bibliothèques peuvent traiter des
modèles avec =\textgreater{} 10\^{}5 variables ou contraintes * Rôle
renctral en Algo car Beaucoup de Problèmes pratiques peuent être
exprimés via ce formalisme

\subparagraph{Le cas des n reines}\label{le-cas-des-n-reines}

Vecteurs inconnu

Z indexe sur les cases de l'echiquier

Z=(Zi,j , i = 1..N, j = 1..N) Entier scrictement entre 0 et 1
(exclues)\ldots{}

Sémantique : Zi,j = 1 \textasciitilde{} on lace une reine en case (i, j)
0 \textasciitilde{} on ne places pas

\textbf{Quantité a maximum (critere qualité)}

\begin{verbatim}
ΣTAB[i,j] * Zi,j
i = 1..N
j = 1..N
\end{verbatim}

\textbf{Contraintes}

\begin{enumerate}
\def\labelenumi{\arabic{enumi}.}
\item
  je veux exactement n reines Σi,j Zi,j = n
\item
  Pas + de 1 reine par ligne Pour toute ligne j = 1..N Σ Zi,j
  =\textless{} 1
\item
  Pas + de 1 reine par colonne Pour toute colonne i = 1..N Σ Zi,j
  =\textless{} 1
\item
  Pas + de 1 reine pour une diagonale montante, 2n-1, diagonales
  montante
\end{enumerate}

Pour tout k = -(n-1)..(n-1)

Σ Zi,i+k =\textless{} 1 i tq 1 \textless{}= i \textless{}= n i tq 1
\textless{}= i+k \textless{}= n

2 types d'algos : - \emph{exact} : Realise exactement la spécification -
\emph{approché} : Induisent une marge d'erreur

Prendre en compte temps d'exec et critère de qualité de l'approximation
dans le cas d'un algo approché (également la robustesse).

Approché

\textbf{Remarque} : Il faut préciser ``appoche'' : - J'impose vraiment
\emph{n-reines} (sinon échec) - J'accepte moins de n-reine → maximiser
l'argent donc pénalités (n - nbres de reines placées)

\begin{verbatim}
not stop;
tant que not stop faire
    choisir une case libre;
    si echec(choisir)
        alors stop;
    sinon
        placer une reine sur la case choisie;
        déduire le plus de choses possibles;
        mettre à jour stop;
\end{verbatim}

→ algo glouton avec propagation de contraintes

Déduire des reines imposées, interdites

\textbf{Exercice} : Écrire algorithme pour réalistion ces déductions
imposé/interdit → SDD, Algo (on ne prend pas en compte le gains ici)

\begin{itemize}
\tightlist
\item
  Un tableau deux dimensions i,j où :

  \begin{itemize}
  \tightlist
  \item
    0 case interdite
  \item
    -1 case libre
  \item
    1 case occupée
  \end{itemize}
\item
  Un tableau compteur ligne, en ligne j nbe de case libre
\item
  statut ligne, statutligne{[}1{]} = 1, il y a une reine en ligne j
\item
  Idem colonnes :

  \begin{itemize}
  \tightlist
  \item
    compteur colonnes
  \item
    statut colonnes
  \end{itemize}
\item
  Liste de reines imposées (fait déclencheurs en propagation de
  contraintes) :

  \begin{itemize}
  \tightlist
  \item
    couples (i,j) correspond à des cases où une reine est imposée
  \item
    ehec, booleen diagnostiquant les impossibiliftés
  \end{itemize}
\end{itemize}

Algo :

\begin{verbatim}
Echec ← 0;
LISTE ← case qi'on vient de choisir dans la boucle principale de l'algo;
Tant que (Not Échec) && (Liste != Nil) Faire
|   (i,j) ← Tête(LISTE);
|   LISTE ← Queue(LISTE);
|   Pour toute case (i',j') telle que (i = i') || (j = j') || [(i+j) = (i'+j')] || [(i-j)=(i'-j')] Faire
|   Si occupé[i',j'] = 1 alors Échec
|   Sinon 
|   |   Si (occupé[i',j'] = -1) alors 
|   |   |   occupé[i',j'] = O;
|   |   |   compteur_lignes[i'] ← compteur_lignes[i]-1
|   |   |   compteur_colonnes[j'] ← compteur_colonnes[j]-1
|   |   |   Si (compteur_lignes[i] = 0) && (statut_lignes[i'] = 0) alors ÉCHEC;
|   |   |   Sinon
|   |   |   |   Si (compteur_lignes[i'] = 1) && (statut_lignes[0] = 0)
|   |   |   |   |   soit j0 ← unique j tq occupé [i',j] = 0;
|   |   |   |   |   occupé[i',j0] ← 1;
|   |   |   |   |   LISTE ← (i',  j0).LISTE;
|   |   |   |   |   statut_lignes[i'] = 1;
|   |   |   |   |   statut_colonnes[j0] = 1;
|   |   |   |   Fin Si
|   |   |   Fin Sinon
|   |   Fin si
|   Fi Sinon
Fin Tant que
\end{verbatim}

\emph{Remarque} : le schéma peut échouer parce qu'il ne trouve pas une
solution estimée acceptable. Il peut aussi renvoyer une solution de
qualité médiocre.

2 pistes d'améliorations :

\begin{enumerate}
\def\labelenumi{\arabic{enumi}.}
\item
  On garde le schéma glouton et on le rend non déterministe
  (``randomize''), de façon à pouvoir l'executer plusieurs fois de
  suite. Fixe le nb de N ``réplications'', pour i = 1..N faire Executer
  l'algo glouton (non déterministe); récuperer le ``meilleur'' résultat
  obtenu; Question : comment est-ce que je rends mon algo ``non
  déterministe'' ? Algo courant (deterministe) Itération courante :
  Cases imposées interdites, Libres (Random avec proba si prob
  supérieure premier sinon deuxieme meilleur)
\item
  À l'issue de l'execution de l'algo glouton, on récupère une solution
  REINE (ex : vecteur indexe sur les cases et à valeurs en {[}0,1{]}) On
  essaie alors d'ameliorer cette solution en lui appliquant des
  opérateurs de ``Transformation locale'' (local search )
\end{enumerate}

(\ldots{})

Schéma GRASP

\begin{verbatim}
Pour i = 1..N (Replication) Faire
    Creer une solution REINE via
    la procédure gloutonne "randomizée";
    not stop;
    tant que not stop;
        Appliquer a reine l'opérateur 0 pour une valeur ad hoc de parametre;
        mettre à jour stop;
    Consinier le meilleur objet REINE obtenu;
\end{verbatim}

Pour mettre en oeuvre ce schema, il me faut définir le (ou les )opératur
0 + la façon d'aleer chercher les bonnes valeurs de paramètres

Ici difficultés (liée au fait qu'il y a beaucoup de contrainessur l'obet
cherché) → j'ai du mal à définir un opératuer 0 qui s'applique à un
placement REINE satisfaisant les contraintes et qui le maintienne dans
les contraines

Question améliorer ?

1ère Approche, replication

2° approche, on applique sur l'objet REINE produit par l'algo glouton
une boucle dite de transformation locale (``local search'') :

\begin{verbatim}
not stop;
tant que not stop faire
    perturber REINE;
\end{verbatim}

Cette 2° approche repose sur le design ``d'operateurs'', c'est à dire de
prcédures \texttt{TRANSFO(Reine,\ λ)} Reine : adresse, λ : valeur.

\textbf{Opérateurs génériques \texttt{Build/Destroy}}

\begin{itemize}
\tightlist
\item
  J'enleve p reines parmi les \emph{n} reines placées (p
  \textasciitilde{} x\% · n)
\item
  Je me retrouve avec q = n-p (ou un peu moins) reines placées;
  J'applique la propagation de contraintes de ces q reines (j'interdis
  et impose des cases)
\item
  Je relance le procede glouton ``randomise'' à partir de la situation
  obtenue
\end{itemize}

l'objet RBIND transforme de ? de ces 3 étapes

L'opérateur BUILD.DESTROY ainsi défini, prend comme paramètre.

\texttt{BUILD.DESTROY(Reine,\ λ)}

\textbf{Questions sous-jacentes}

\begin{enumerate}
\def\labelenumi{\arabic{enumi}.}
\tightlist
\item
  Comment je choisis p et les reines de la liste λ ?
\item
  Qu'est ce qu'on met derriere ``Tant que not stop faire perturber
  (Reine)'' ?
\end{enumerate}

\emph{Question 1}

\begin{verbatim}
Fixer les idées →
    n = 100
    L'objet REINE est un vecteur a taille 10000
    ➡ À chaque etape p = 20 (20%) → on enlève 20 reines
    Pas possibilité d'enumerer toutes les possibilités
\end{verbatim}

\begin{enumerate}
\def\labelenumi{\arabic{enumi}.}
\tightlist
\item
  Possibilité : Enlever les reines qui ont été placées en 1°
\item
  Possibilité : Enlver les reines faibles (qui portent le moins
  d'argent)
\end{enumerate}

\textbf{NB :} Il est souhaitable de génerer plusieurs paquets de p
reines à faire a les tester tous et selectionner le + approprié.

Je peux mixer les 2 critères

Spécification de la boucle ``local search''

\begin{verbatim}
not stop; solcour ← reine;
tant que not stop faire
    not stop 1;
    Tant que not stop 1 faire
        generer un paquet λ de reines à enlever; //  (on peut générer eventuellement tous les paquets des les début)
        Tester l'application de BUILD.DESTROY(REINE, λ); // Utilise une copie de REINE
        Si OK(Tester) alors
            stop1;
            Appliquer BUILD.DESTROY(Reine, λ);
        Sinon mettre à jour stop1;
\end{verbatim}

\emph{Reste à préciser}

\texttt{OK(Tester)} \textasciitilde{} Dans quelles conditions j'estime
que le paquet λ de reines à retirer justifie l'application de
BUILD.DESTROY(Reine, λ)

\begin{enumerate}
\def\labelenumi{\arabic{enumi}.}
\item
  Approche : OK si l'application de l'operateur ameliore REINE (place
  plus de reine ou fait gagner + d'argent) (\emph{Descente ou
  Hill-climbing})
\item
  Approche : OK toujours vrai → je genere 1 paquet et j'applique
  (\emph{Marche aléatoire ou Random Walk})
\end{enumerate}

→ Approche mixte : Recul simulé, Tabou, Genetique

(\ldots{})

\textbf{Rappel}

Exploration en Arbre (méthode exacte) → en largeur, liste de noeud créés
: pb + décisions prises, placement de reines ; liste de triplés (ε(de
signe +/-),i,j)

\emph{ex} : Imposé reine en case (2,3) Interdire en (5,4) →
\{(+,2,3),(-,5,4)\}

Variable de controle de l'exploration en largeur de l'arbre : LISTE :
Liste de noeuds

\emph{ex} : LIST : \{\{+,2,3\},\{+,5,4\}\} ← Noeud 1,
\{(+,2,3),(-5,4),\{(-,2,3)\}\}

\textbf{Squelette de l'Algo}

\begin{verbatim}
*SDD* : LISTE,
                REINE_COUR, 
                VAL_COUR, (meilleurs objets obtenus)
                OCCUPÉ[n][n] : de 1..n, 0/1/-1,
                libre_col, libre_lig, statut_col, statut_lig

Initialisation
    OCCUPÉ ← 1;
    ------
    LISTE ← {nil}; not stop;
    REINE_COUR ← Indéfini;
    VAL_COUR ← -∞

Corps Algorithme
    Tant que (not stop) && (LISTE ≠ nil) faire
        N ← Tete(LISTE);  // Liste  de cases imposées / interdites
        LISTE ← Queue(LISTE);
        Remplir OCCUPÉ, libre_col, ... via l'algo de déduction (prop de contraintes)

        Choisir une case (i0,j0) libre
        Généer les 2 noeuds 
            n1 ← (+,i0,j0) · N; n2 ← (-,i0,j0);
            En utilisant une copie a OCCUPÉ, appliquer le precessus de déduction a la décsion (+i0,j0);
            Si succes, mois,aucune case de libre, on a une vraie solution et on cose sa valeur = valeur : si cette valeur > valcour, on met à jour REINE_COUR et VAL_COUR; sinon (et si succes), on met N1, dans LISTE;
            Idem avec N2;
\end{verbatim}

\textbf{Principe} : variable la plus contrainte

\emph{Question} : ``Mettre n1 ou n2 dans LISTE''

Est-ce qu'il y a un ordre pour les éléments de LISTE ?

\emph{Réponse} : On va essayer de noter les noeuds, à l'aide d'un
procédé d'estimation optimiste. (Banch / Bound) → LISTE sera alors
ordonnée par notes décroissantes

Je calcule une valeur VAL en tenant compte de de certaines contraintes
``faciles'' et en delaissant les contraintes ``difficiles'' (lignes,
diagonales)

VAL \textgreater{}= La valeur du meilleur placement compatible ? le
noeud ( Estimation optimiste)

\textbf{Mecanismes de filtrage induits}

\begin{itemize}
\tightlist
\item
  Règle destructive VAL \textless{}= VALCOUR \textbar{}= Couper
  \textasciitilde{} j'élimine le noeud
\item
  Règle constructive Si la solution associée à VAL satisfait les
  contraintes difficiles, aors elles se ??
\end{itemize}

Adaptation de l'algorithme REINE à l'utilisation de Estimation optimiste

LISTE devient une liste de couples (noeud, valeur) 1. Reste identique 2.
Si n1 (n2) débouche sur un succès et des cases libres après propagation
→ je calcule la note VAL1 de n1 (idem pour N2) et j'applique le me
filtrage : * Si v1 \textless{}= VALCOUR alors Exit n1 * si v1
\textgreater{} VALCOUR, ET le placement induit satisfait toutes les
contraintes → Je mets à jour REINE\_COUR et Exit n1; * Sinon j'insère
(n1, val1) dans liste de façon à garder liste ordonné pour VAl1
décroissant (idem n2)

STOP ? quand la note cal du 1e element dans liste est \textless{}=
VALCOUR

\section{\texorpdfstring{Problème orienté
``simulation''}{Problème orienté simulation}}\label{probluxe8me-orientuxe9-simulation}

A,B,C postes de travail

Produits (A), (B)

Produit (A), (B) arrivent ``aléatoirement'' → L\_A, L\_B, \emph{lois
d'arrivées}

On connait des durées de traitement pour (A), (B) sur A,B,C T\_A, T\_B,
T\^{}C\_A, T\^{}C\_B aléatoires

→ On se fait une idée : - Taux de perte - durée d'attente - Taille des
buffers à augmenter ?

\emph{NB} : Ce n'est pas un poste d'assemblage, mais une machine de
transformation

\textbf{Questions à poser} * Est-ce qu'on a observé des
dysfonctionnements ? * Quels sont les enjeux économiques de ces
dysfonctionnements ? * Données quatifiées * stock : S\_A = 10 ; S\_B = 7
; S\_C = 13 * Dans la simulation faut-il prévoir (Ici non pour tout car
pas réel) * Maintenance préventive * set-up (màj de la machine chaque
fois qu'elle change de tâche) * Relais liés aux changements d'activités
humaines * \emph{Quantifier} les lois L\_A, L\_B d'arrivées des objets A
et B

\textbf{Plusieurs options} - Rythme d'arrivée presque
deterministe\ldots{} - Rythme d'arrivée avec grosses variations *
L'intervalle de temps devient une \emph{variable aléatoire} avec sa
moyenne, variance, sa loi\ldots{} - Il peut y avoir des correlations
entre ces intervalles, et des correlations entre ces intervalles et les
instants. (compliqués)

Avec une loi
\texttt{L\_A\ \textasciitilde{}\ Proba\ qu\textquotesingle{}un\ objet\ A\ arrive\ entre\ 2\ instants\ t\ et\ t+1\ \textasciitilde{}\ O·3\ \ \ \ \ \ \ \ \ L\_B\ \textasciitilde{}\ 0·2}
* Idem pour les durées =\textgreater{} on va supposier ici T\_A
\textasciitilde{} Proba que l'objet A étant strictement à l'instant t,
il soit fini à l'instant t+1 = 0·4\ldots{}

\end{document}
