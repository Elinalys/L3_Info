% simple.tex – Un exemple d'article simple pour illustrer la structure d'un document.
\documentclass[12pt]{article}
\usepackage[scale=0.85]{geometry}
\usepackage[french]{babel}
\usepackage[T1]{fontenc}
\usepackage[utf8]{inputenc}

\setlength\parindent{0pt}
% \date{}
\date{\vspace{-5ex}}

\author{\vspace{-5ex}}
\title{\textsc{TD2 - Probabilités et statistiques}\\}
\begin{document}

\maketitle
% \dotfill
% \rule{\textwidth}{0.1pt}
% \rule{\textwidth}{1.0pt}

\paragraph*{Exercice 1} Quantité de sulfate contenue dans l'air (en $\mu/m^3$) en novembre à Brest au cours du mois de novembre des dix dernières années.

$$10.83\ \ 8.90\ \ 14.71\ \ 12.35\ \ 11.86\ \ 13.80\ \ 11.75\ \ 9.68\ \ 9.33$$ 

On suppose que ces mesures sont des réalisations d'une variables aléatoire suivant une loi normale.


\begin{enumerate}
\item \mu \ val minimale avec confiance \alpha = 99\%. \mu \in ] \bar{x} - \frac{^\sigma}{}



\item 

\end{enumerate}

\end{document}

