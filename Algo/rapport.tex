\documentclass[12pt, a4paper]{report}

%====================== PACKAGES ======================
% \usepackage{extsizes}
\usepackage{tikz}
\usepackage[french]{babel}
\usepackage[utf8x]{inputenc}
\usepackage{hyperref}
\usepackage{multicol}
\usepackage[scale=0.8]{geometry}
\usepackage{tkz-graph}
\usepackage{pgf,tikz}
\usepackage{graphicx}
% \usepackage{fontspec}

\newenvironment{Figure}
  {\par\medskip\noindent\minipage{\linewidth}}
  {\endminipage\par\medskip}

\renewcommand{\baselinestretch}{1.25} 

\usepackage{caption}

\hypersetup{
    % bookmarks=true,         % show bookmarks bar?
    unicode=false,          % non-Latin characters in Acrobat’s bookmarks
    pdftoolbar=true,        % show Acrobat’s toolbar?
    pdfmenubar=true,        % show Acrobat’s menu?
    pdffitwindow=false,     % window fit to page when opened
    pdfstartview={FitH},    % fits the width of the page to the window
    pdftitle={Rapport},    % title
    pdfauthor={Loris},     % author
    pdfsubject={Prepro},   % subject of the document
    pdfcreator={},   % creator of the document
    pdfproducer={}, % producer of the document
    pdfkeywords={}, % list of keywords
    pdfnewwindow=true,      % links in new PDF window
    colorlinks=true,       % false: boxed links; true: colored links
    linkcolor=black,          % color of internal links (change box color with linkbordercolor)
    linkbordercolor=white,
    citecolor=green,        % color of links to bibliography
    filecolor=magenta,      % color of file links
    urlcolor=cyan           % color of external links
}

\usepackage[T1]{fontenc}

\author{Loris \textsc{Croce}}

\title{\rule{\textwidth}{1pt} \\ \Huge\textsc{Rapport de TP d'Algorithmique} \\ \rule{\textwidth}{1pt}}

\begin{document}

\maketitle{}

\tableofcontents

% ajouter manuel build file etc. compilation / execution

\abstract{Pour ce TP le langage \textbf{Java} a été utilisé. Pour la compilation et l'éxecution, le logiciel \textbf{Ant} a été utilisé en utilisant des fichier de \emph{build}. Pour compiler le projet (Ant doit être installé au préalable) il faut se placer dans le répertoire du TP correspondant et taper la commande : \texttt{ant}, pour l’exécuter il faudra entrer : \texttt{ant run.}

\chapter{Gestion de partitions}

    \section{TDA Gestion de partition}

    Le TDA \emph{Gestion de partition} est un TDA qui qui définit une partition $p$ composée d'éléments $e$\footnote{Ici, par souci de simplicité les éléments seront des entiers}, lesquels sont regroupés en son sein par classes. Il dispose de deux opérations :
    \begin{itemize}
        \item \texttt{p.classe(e)} : retourne la classe de l'élément $e$.
        \item \texttt{p.fusion(c1, c2)} : fusionne les classes $c_1$ et $c_2$ de la partition $p$.
    \end{itemize}

    % \subsection{Implémentation}

    Pour implémenter ce TDA une \emph{interface} \texttt{GestionPartition} qui définit la structure du TDA a été créée. Deux solutions ont été implémentées : Le \emph{Tableau de partition} et le \emph{Tableau de pères}.

    \subsection{Tableau de partition}

    Comme les éléments gérés ici sont des entiers, il est possible de les représenter sous la forme d'indices d'un tableau où les cases contiendront la classe dans laquelle chaque élément est stocké. La classe associée à cette implémentation est \texttt{TableauPartition}.

    Les complexités associées à cette implémentation sont :
    \begin{itemize}
        \item \texttt{p.classe(e)} : $\mathcal{O}(1)$, ici implémentée par \texttt{getClasse(int e)}.
        \item \texttt{p.fusion(c1, c2)} : $\mathcal{O}(n)$, ici implémentée par \texttt{fusion(int c1, int c2)}.
    \end{itemize}

    \subsection{Tableau de pères}

    Cette implémentation est associée à celle par \emph{forêts} en la traduisant sous forme de tableau où comme pour le tableau de père les indices représentent les éléments mais les valeurs des cases renseignent le \emph{père} de l'élément\footnote{si cet élément est racine on considèrera qu'il est son propre père.}. La classe associée à cette implémentation est \texttt{TableauPerePartition}.

    Les complexités associées à cette implémentation sont :
    \begin{itemize}
        \item \texttt{p.classe(e)} : $\mathcal{O}(h)$, ici implémentée par \texttt{getClasse(int e)}.
        \item \texttt{p.fusion(c1, c2)} : $\mathcal{O}(1)$, ici implémentée par \texttt{fusion(int c1, int c2)}.
    \end{itemize}

    \section{Composantes connexes}

    Pour résoudre le problème des composantes connexes il a été nécessaire de créer une classe \texttt{Graphe} contenant les arêtes sous la forme d'une matrice d'entiers et les sommets sous la forme d'un tableau d'entiers. Il a également fallu surcharger le constructeur de \texttt{TableauPartition} pour qu'il prenne en paramètre un type \texttt{Graphe} et effectue les fusions nécessaires pour que les sommets reliés appartiennent à la même classe, montrant ainsi les composantes connexes du graphe.

    \paragraph{Exemple}

    Un graphe $G$ de la forme :

    \begin{figure}[h]
        \centering
        \begin{tikzpicture}[scale=3]
            \GraphInit[vstyle=Normal]
            \Vertex[x=0,y=1]{0}
            \Vertex[x=1,y=1]{1}
            \Vertex[x=0,y=0]{2}
            \Vertex[x=1,y=0]{3}
            \tikzset{EdgeStyle/.style={-}}
            % \Edge[style={bend right}](0)(2)
            \Edge(0)(2)
            \Edge(1)(3)
            % \Edge[style={bend left}](1)(3)
            % \Edge[label=$5$](1)(3)
            % \Edge(1)(3)
        \end{tikzpicture}
        % \caption{États d'un processus}
    \end{figure}

    Donnera un tableau de la forme :

    \begin{table}[h]
    \centering
    \label{my-label}
    \begin{tabular}{|l|l|l|l|l|}
    \hline
    \textbf{Valeurs} & 0 & 1 & 2 & 3 \\ \hline
    \textbf{Classes} & 0 & 1 & 0 & 1 \\ \hline
    \end{tabular}
    \end{table}

    \section{Arbres couvrants minimaux}

    (\dots)

\chapter[Programmation dynamique. Distance d'édition]{Programmation dynamique {\ \normalfont\emph{Distance d'édition}}}

    Ici, on va chercher à implémenter l'algorithme de \textbf{distance d'édition} qui consiste à rechercher le nombre d'opérations --\emph{insertion}, \emph{suppression} ou \emph{substitution} de caractères-- nécessaires pour passer d'un mot donné à un autre. On définit donc une classe \texttt{Comparaison} qui contiendra toutes les méthodes associées.

    \section{Méthode récursive}

    Pour cette méthode on créée donc une méthode \texttt{Recursif} avec pour paramètre deux chaines de caractères $A$ et $B$. On définit deux cas d'arrêts :
    \begin{itemize}
        \item $A$ est nul.
        \item $B$ est nul.
    \end{itemize}
    Sinon on rappelle l'algorithme sur des sous parties de $A$ et $B$ en incrémentant la distance si les deux caractères sélectionnés sont différents.

    \section{Avec mémoïzation}

    La méthode récursive avec \emph{mémoïzation} reprend le même principe en ajoutant un tableau de distances entre sous-chaînes où les valeurs sont ici initialisées à $999$\footnote{Pour simuler un $+\infty$.} et un \emph{niveau} qui s'incrémentera si l'on repère une différence de caractère dans l'appel récursif.

    \section{Méthode dynamique}

    Pour l'algorithme de programmation dynamique les chaines de caractères placées en paramètres sont remplacées par des \emph{tableaux de caractères} (comme en C), pour effectuer la transformation à l'appel de la fonction on appelle la fonction \texttt{toCharArray()} du type \texttt{String} de Java. La fonction \texttt{Dynamique} utilise une matrice à deux dimensions qui stocke les résultats intermédiaires pour calculer les suivants et la retourne.

    \paragraph{Exemple} avec les mots \emph{tourte} et \emph{tartre}

    \begin{table}[h]
    \centering
    % \caption{My caption}
    % \label{my-label}
    \begin{tabular}{|c|c|c|c|c|c|c|c|}
    \hline
    \textbf{} & \textbf{$\epsilon$} & \textbf{t} & \textbf{a} & \textbf{r} & \textbf{t} & \textbf{r} & \textbf{e} \\ \hline
    \textbf{$\epsilon$} & 0 & 1 & 2 & 3 & 4 & 5 & 6 \\ \hline
    \textbf{t} & 1 & 0 & 1 & 2 & 3 & 4 & 5 \\ \hline
    \textbf{o} & 2 & 1 & 1 & 2 & 3 & 4 & 5 \\ \hline
    \textbf{u} & 3 & 2 & 2 & 2 & 3 & 4 & 5 \\ \hline
    \textbf{r} & 4 & 3 & 3 & 2 & 3 & 3 & 4 \\ \hline
    \textbf{t} & 5 & 4 & 4 & 3 & 2 & 3 & 4 \\ \hline
    \textbf{e} & 6 & 5 & 5 & 4 & 3 & 3 & 3 \\ \hline
    \end{tabular}
    \end{table}

    \section{Affichage des opérations}

    \section{Comparaison des résultats}
            
\end{document}