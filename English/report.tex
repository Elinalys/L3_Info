\documentclass[12pt, a4paper]{article}

%====================== PACKAGES ======================
\usepackage{extsizes}
\usepackage{tikz}
% \usepackage[french]{babel}
\usepackage[utf8x]{inputenc}
\usepackage{hyperref}
\usepackage{multicol}
\usepackage[scale=0.8]{geometry}
% \usepackage{fontspec}

\newenvironment{Figure}
  {\par\medskip\noindent\minipage{\linewidth}}
  {\endminipage\par\medskip}

\renewcommand{\baselinestretch}{1.25} 

\usepackage{caption}

\hypersetup{
    % bookmarks=true,         % show bookmarks bar?
    unicode=false,          % non-Latin characters in Acrobat’s bookmarks
    pdftoolbar=true,        % show Acrobat’s toolbar?
    pdfmenubar=true,        % show Acrobat’s menu?
    pdffitwindow=false,     % window fit to page when opened
    pdfstartview={FitH},    % fits the width of the page to the window
    pdftitle={Rapport},    % title
    pdfauthor={Loris},     % author
    pdfsubject={Prepro},   % subject of the document
    pdfcreator={},   % creator of the document
    pdfproducer={}, % producer of the document
    pdfkeywords={}, % list of keywords
    pdfnewwindow=true,      % links in new PDF window
    colorlinks=true,       % false: boxed links; true: colored links
    linkcolor=black,          % color of internal links (change box color with linkbordercolor)
    linkbordercolor=white,
    citecolor=green,        % color of links to bibliography
    filecolor=magenta,      % color of file links
    urlcolor=cyan           % color of external links
}

\usepackage[T1]{fontenc}

\author{Elouan \textsc{Lebaillif} \and Jordan \textsc{Daffix} \and Loris \textsc{Croce}}

\title{\rule{\textwidth}{1pt} \\ \Huge\textbf{IT Sounds cool : } \\ \emph{English L3} \rule{\textwidth}{1pt}}

\begin{document}

\maketitle{}

\tableofcontents

\section{Introduction}

We are Jordan, Loris and Elouan, students in the third degree in Computer Sciences, at the University Clermont Auvergne.

\paragraph{Why this project}
We thought some people were boring so we decided to create an headset that could mute them IRL, thats was the first idea...

\section{Description of the project}

Role of each team member :

\begin{itemize}
    \item Rapport (mise-en-page Loris, rédaction tlm ?)
    \item Montage vidéo (Jordan)
    \item Recherche vidéo (tlm)
    \item Script (Jordan)
    \item Images application (...?)
    \item Idea  (Jordan + améliorations autres ?)
    \item Logo (Elouan)
    \item Maquette (Elouan)
\end{itemize}

\section{Description of the product}

    \subsection{Technical description}
        - casque qui filtre les sons
        - exemples (prof, )

    \subsection{Physical description}

    images du produit
colors, dimensions, weight, battery, 
images de l’application
=> “Daredevil” vue pour supprimer certaines sources environnantes visibles
=> Presets

    \subsection{Functional description}
    Functional description
image explication filtering
Works with an external application allowing to custom filters and install some additional ones(\$) 

\section{Script}

Résumé : la vidéo nous montre tout d’abord les sources d’énervements quotidiennes des gens (“n’avez-vous jamais etc.”) ; “Nos scientifiques ont travaillé etc.” (vidéo scientifique + maths) ; “voici notre casque blabla” (images du produit) ; explication comment ça marche ;
bonheur (vidéo soleil entrepreneuse + vélo ou champ… => liberté) ; slogan.

Vrai script :

N’av

slogan : I.T. Sounds Cool ; (it) sounds cool.
(ou) :  I.T. Sounds Cool ; sounds cool, isn’t it ?

Sum up : The video shows us first the annoying noises of everyday life (“haven’t you …”). “Our scientists worked…” (scientific video + math) ; “Here’s our new product” (products images) ; How it works, explanations; Happiness (sunny happy people) ; slogan.


(vidéo scientifique + vidéo maths)
Nos scientifiques ont travaillé sur...

exemples sources à supprimer :
Vidéo OK : prof / bébé / tapotage
A trouver : mauvaise musique /  personne étudiant / relou


% Pas oublier pages perso

\end{document}